\phantomsection
\chapter*{Homework Assignments}
\addcontentsline{toc}{chapter}{Homework Assignments}
\markboth{Homework Assignments}{Homework Assignments}

\phantomsection
\section*{Assignment 1}
\addcontentsline{toc}{section}{Assignment 1}

\begin{enumerate}[label={\sffamily\bfseries\color{main}\arabic*.}]
	\item Let $G$ be a group and $H$ be a subgroup of $G$. For any $g\in G$, prove that
	$$gHg^{-1}=\left\{ghg^{-1}\mid h\in H\right\}$$
	is also a subgroup of $G$.

	\item Determine all the non-isomorphic abelian groups of order 2024. Justify your answer.

	{\footnotesize\textit{\color{main}Hint.} $2024=2^3\cdot 11\cdot 23$.\par}

	\item Let $p$ and $q$ be distinct primes. Prove that the number of distinct finite abelian groups of order $p^4$ is the same as the number of distinct finite abelian groups of order $q^4$. How many distinct finite abelian groups of order $p_1^4p_2^4\cdots p_r^4$ are there if $p_1,\hdots,p_r$ are all distinct primes?

	\item You are given a finite abelian group with $|G|=16$. As well, you are told that $G$ has an element of order 8 and two elements of order 2. What group is $G$ isomorphic to? Justify your answer.

	\item Let $G$ be an abelian group where the operation is addition. Let $K$ be a proper subgroup of $G$, and suppose that $d\in G\setminus K$, that is $d$ is an element of $G$ not in $K$. Show that the set
	$$H=\{k+zd\mid k\in K,z\in\mathbb Z\}$$
	is also a subgroup of $G$ with $K\subsetneq H$.

	{\footnotesize\textit{\color{main}Remark.} This exercise justifies a step we used in the proof of the Fundamental Theorem of Finite Abelian Groups given in Lecture 5.\par}

	\item Suppose $G$ is an abelian group of order 25. You are able to ask an ``oracle'' about the order of a particular element in the group. What is the maximum number of times you have to ask the ``oracle'' for an answer to figure out the structure of $G$? Justify your answer.

	\item Find all composition series of $S_3\times\mathbb Z_3$.

	\item Let $G$ and $H$ be solvable groups. Show that $G\times H$ is also solvable.
\end{enumerate}

\vfill\pagebreak

\phantomsection
\section*{Assignment 2}
\addcontentsline{toc}{section}{Assignment 2}

\begin{enumerate}[label={\sffamily\bfseries\color{main}\arabic*.}]
	\item Show that $D_{2024}$ is a solvable group (for clarity, this is the dihedral group on the 2024-gon, so this group has 4048 elements).

	\item Let $N$ be a normal subgroup of $G$. If $N$ and $G/N$ are solvable groups, show that $G$ is a solvable group.

	\item Let $X=\{1,2,3,4,5,6\}$ and consider the group $H=\{(1),(1\, 4)(2\, 5)(3\, 6)\}$. The elements of $H$ act on $X$ as functions. Determine all the unique orbits of this action and write $X$ as a partition of these orbits.

	\item Find the class equation for $D_5$. Show all your work.

	\item A flag with seven horizontal stripes can be coloured with three different colours. How many distinct flags can you make?

	\item What does the First Sylow Theorem tell you about all the groups of order 2024? What does the Third Sylow Theorem tell you about the Sylow 23-subgroups of a group of order 2024?

	\item Prove that a noncyclic group of order 21 must have 14 elements of order 3.

	{\footnotesize\textit{\color{main}Hint.} The theorem given below will be helpful.\par}

	\begin{theorem*}
		Let $G$ be a finite group and suppose that $M$ and $N$ are normal subgroups of $G$ such that $M\cap N=\{e\}$ and $|M||N|=|G|$. Then $G\cong M\times N$.
	\end{theorem*}

	\item Prove that if $G$ is a group with $|G|=99$, then $G\cong\mathbb Z_{99}$ or $G\cong\mathbb Z_3\times\mathbb Z_{33}$.

	{\footnotesize\textit{\color{main}Hint.} The theorem given previously will be helpful.\par}

	\item Go to \url{http://abstract.ups.edu/aata/aata.html} and review the tutorial in Chapter 14. Also, review the Sage documentation found here:

	{\centering\footnotesize\url{https://doc.sagemath.org/html/en/thematic_tutorials/group_theory.html#conjugacy}\par}

	Now find the class equations for $D_3, D_4, \hdots , D_{10}$. (Note, you will be able to check your answer for Exercise 4).
\end{enumerate}

\vfill\pagebreak

\phantomsection
\section*{Assignment 3}
\addcontentsline{toc}{section}{Assignment 3}

\begin{enumerate}[label={\sffamily\bfseries\color{main}\arabic*.}]
	\item Let $\vec x=\begin{bmatrix}
	x_1 & x_2 & x_3
	\end{bmatrix}^\mathsf{T}$ be a point on the unit sphere in $\mathbb R^3$. That is, ${x_1}^2+{x_2}^2+{x_3}^2=1$. Prove that if $A\in\mathrm{O}_3(\mathbb R)$, where $\mathrm{O}_3(\mathbb R)$ is the group of $3\times 3$ orthogonal matrices, then $A\vec x$ is also on the unit sphere.

	\item Let $\varphi\colon R\to S$ be a homomorphism of rings. If $J$ is an ideal of $S$ and $I=\{r\in R\mid \varphi(r)\in J\}$, then prove that $I$ is an ideal of $R$ and $\ker(\varphi)\subseteq I$.

	\item Let $T$ be the set of rational numbers whose denominators (in lowest terms) are not divisible by 101 (which is a prime number). Prove that $T$ is a subring of $\mathbb Q$.

	\item \textit{This question extends from the previous question.}

	Let $I$ be the elements of $T$ such that $101$ divides the numerator of an element of $T$. Prove that $I$ is an ideal of $T$ and that $T/I\cong\mathbb Z_{101}$.

	{\footnotesize\textit{\color{main}Hint.} Recall that if $101\!\nmid\! s$, then $s\not\equiv 0\pmod{101}$. Furthermore, since $\mathbb Z_{101}$ is a field, there is a $u\in\mathbb Z_{101}$ such that $su=1$ in $\mathbb Z_{101}$.\par}

	\item Prove or disprove the following statements:
	\begin{enumerate}
		\item If $R$ is a commutative ring, then $R[x]$ is a commutative ring.
		\item If $R$ has an identity, then $R[x]$ has an identity.
		\item If $R$ is a field, then $R[x]$ is a field. 
	\end{enumerate}

	\item Consider the \textbf{derivative map} $\D\colon\mathbb R[x]\to\mathbb R[x]$ given by
	$$\D(a_0+a_1x+a_2x^2+\cdots +a_nx^n)=a_1+2a_2x+\cdots +na_nx^{n-1}.$$
	Is $\D$ a ring homomorphism? Either prove this statement or give a counterexample.

	\item Recall that a nonzero element $a\in R$ is \textbf{nilpotent} if there is a positive integer $k\geq 2$ such that $a^k=0$. Suppose that $a_0$ is a unit and $a_1$ is a nilpotent element of $R$. Prove that $a_0+a_1x$ is a unit in $R[x]$.

	\item Suppose $R$ is an integral domain. Assume that the division algorithm holds in $R[x]$. Prove that $R$ is a field.

	\item Go to \url{http://abstract.ups.edu/aata/aata.html} and review the tutorial in Chapter 17. Also look for documentation on the SAGE command \verb|quo_rem|. Use SAGE to answer all of the problems of Exercise 3 of Section 17.5.
\end{enumerate}

\vfill\pagebreak

\phantomsection
\section*{Assignment 4}
\addcontentsline{toc}{section}{Assignment 4}

\begin{enumerate}[label={\sffamily\bfseries\color{main}\arabic*.}]
	\item Prove the Rational Root Theorem.

	\begin{theorem*}[Rational Root Theorem]
		Let
		$$p(x)=a_nx^n+a_{n-1}x^{n-1}+\cdots+a_0\in\mathbb Z[x]$$
		with $a_n\neq 0$. If $\frac rs$ is a rational number with $\gcd(r,s)=1$ such that $p\left(\frac rs\right)=0$, then $r$ divides $a_0$ and $s$ divides $a_n$.
	\end{theorem*}

	\item If $f(x)=a_nx^n+\cdots +a_0$ is a polynomial in $\mathbb Z[x]$ and if $p$ is a prime that does not divide $a_n$, we can consider the polynomial $\bar f(x)=[a_n]x^n+\cdots+[a_0]\in\mathbb Z_p[x]$, where $[a_i]$ denotes the equivalence class of $a_i$ in $\mathbb Z_p$. It can be shown that if $\bar f(x)$ is irreducible in $\mathbb Z_p$, then $f(x)$ is irreducible in $\mathbb Q[x]$. Use this fact to show the following polynomials are irreducible in $\mathbb Q[x]$:
	\begin{enumerate}
		\item $7x^3+6x^2+4x+6$
		\item $9x^4+4x^3-3x+7$
	\end{enumerate}

	{\footnotesize\textit{\color{main}Remark.} The proof of this fact can be found in most abstract algebra textbooks. It gives you another tool to check if a polynomial is irreducible.\par}

	\item Consider the following subring of $\mathbb Q$ that is also a domain:
	$$R=\left.\left\{\frac{n}{2^i}~\right|~ n\in\mathbb Z,i\geq 0\right\}.$$
	Prove that the field of fractions $F_R$ is isomorphic to $\mathbb Q$.

	{\footnotesize\textit{\color{main}Remark.} In the above result, we can replace 2 with any prime $p$ and get a similar result. Consequently, there are an infinite number of domains $R$ with $\mathbb Z\subseteq R\subseteq\mathbb Q$ whose field of fractions is isomorphic to $\mathbb Q$.\par}

	\item Let $D$ be a PID. Prove that every ideal of $D$ is contained in a maximal ideal.

	\item Let $D$ be a Euclidean Domain with corresponding Euclidean valuation $v$. Prove that $u\in D$ is a unit if and only if $v(u)=v(1)$.

	\item The ring $\mathbb Z[i]=\{a+bi\mid a,b\in\mathbb Z\}$ with $i^2=-1$ is a Euclidean Domain via the Euclidean valuation $v(a+bi)=a^2+b^2$.
	\begin{enumerate}
		\item Find all the units of $\mathbb Z[i]$.
		\item Show that if $v(a+bi)$ is a prime number, then the element $a+bi$ is an irreducible element of $\mathbb Z[i]$.

		{\footnotesize\textit{\color{main}Hint.} The previous question may be helpful.\par}
	\end{enumerate}

	\item The ring $\mathbb Z[i]=\{a+bi\mid a,b\in\mathbb Z\}$ with $i^2=-1$ is a Euclidean Domain via the Euclidean valuation $v(a+bi)=a^2+b^2$. Find $q$ and $r$ such that $2024+i=(1+2024i)q+r$ with $r=0$ or $v(r)<v(1+2024i)$. In other words, apply the division algorithm to $z=2024+i$ and $w=1+2024i$.

	\item A ring $R$ has the \textbf{descending chain condition} if for every descending chain of ideals of $R$
	$$I_1\supseteq I_2\supseteq I_3\supseteq \cdots,$$
	there exists an integer $N$ such that $I_N=I_{N+1}=I_{N+2}=\cdots$.
	\begin{enumerate}
		\item Show that $\mathbb Q[x]$ does not have the descending chain condition.
		\item Prove that an integral domain $R$ is a field if and only if $R$ satisfies the descending chain condition.
	\end{enumerate}

	{\footnotesize\textit{\color{main}Hint.} If $a\in R$ is such that $a\neq 0$ and $a$ is not a unit, what can be said about the chain of ideals $(a)\supseteq(a^2)\supseteq(a^3)\supseteq\cdots$?\par}
\end{enumerate}

\vfill\pagebreak

\phantomsection
\section*{Assignment 5}
\addcontentsline{toc}{section}{Assignment 5}

\begin{enumerate}[label={\sffamily\bfseries\color{main}\arabic*.}]
	\item Suppose that $p(x)$ is an irreducible polynomial of degree 2024 in $\mathbb Z_2[x]$. How many elements are in the field $\mathbb Z_2[x]/(p(x))$? How does your answer change if $p(x)$ is an irreducible polynomial in $\mathbb Z_p[x]$ with $p$ a prime?

	\item Show that $\sqrt{1+\sqrt{1+\sqrt{2022}}}$ is algebraic over $\mathbb Q$. What is the minimal polynomial of this element?

	\item If $r$ and $s$ are nonzero integers, prove that $\mathbb Q(\sqrt{r})=\mathbb Q(\sqrt s)$ if and only if $r=t^2s$ for some $t\in\mathbb Q$.

	\item Show that $\mathbb C$ is algebraic over $\mathbb R$.

	\item Let $\alpha$ be an algebraic element of $E$ over $F$ whose minimal polynomial in $F[x]$ has odd degree. Prove that $F(\alpha)=F(\alpha^2)$.

	{\footnotesize\textit{\color{main}Hint.} Verify that $F(\alpha,\alpha^2)=[F(\alpha^2)](\alpha)=F(\alpha)$.\par}

	\item Let $n_1,\hdots,n_t$ be $t$ distinct positive integers. Prove that
	$$[\mathbb Q(\sqrt{n_1},\hdots,\sqrt{n_t}):\mathbb Q]\leq 2^t.$$

	\item Compute a basis for the extension $\mathbb Q(\sqrt{2024},i)$ over $\mathbb Q$. What is $[\mathbb Q(\sqrt{2024},i):\mathbb Q]$?

	\item Prove or disprove: $\mathbb Q(\sqrt 5)\cong\mathbb Q(\sqrt 2)$.

	\item Prove that the following three statements are equivalent:
	\begin{enumerate}
		\item $F$ is an algebraically closed field.
		\item Every irreducible polynomial in $F[x]$ has degree 1.
		\item Every non-constant polynomial in $F[x]$ splits in $F$.
	\end{enumerate}

	\item Is it possible to construct with a straightedge and compass an isosceles triangle of perimeter 8 and area 1?

	{\footnotesize\textit{\color{main}Hint.} No. You may want to use Exercise 2 of Assignment 4 (I used it in my solution).\par}
\end{enumerate}