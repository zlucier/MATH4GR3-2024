\section{Lecture 1 — Groups and Basic Definitions}

\begin{definition}[group]
	A \textbf{group} $G$ is a set with a binary operation $*$ such that
	\begin{enumerate}
		\item There is an element $e\in G$ such that $a*e=e*a=a$ for all $a\in G$ \textit{(identity)};
		\item $a*(b*c)=(a*b)*c$ for all $a,b,c\in G$ \textit{(associativity)};
		\item For each element $a\in G$ there is an element $a^{-1}\in G$ such that $a*a^{-1}=e$ \textit{(inverse)}.
	\end{enumerate}
	The group $G$ is said to be \textbf{abelian} if $a*b=b*a$ for all $a,b\in G$.
\end{definition}

\begin{definition}[group order]
	The \textbf{order} of a group $G$ is simply its cardinality $|G|$. We say that $G$ is a \textbf{finite} group if $|G|<\infty$.
\end{definition}

\begin{proposition}
	Let $G$ be a group. Then
	\begin{enumerate}
		\item The identity $e\in G$ is unique;
		\item For each $a\in G$ the inverse $a^{-1}$ is unique;
		\item If $a*b=a*c$, then $b=c$;
		\item $(a^{-1})^{-1}=a$;
		\item $(a*b)^{-1}=b^{-1}*a^{-1}$.
	\end{enumerate}
\end{proposition}

We will often omit the group operation symbol $(*)$ if the operation is clear. That is, we will denote $a*b$ by $ab$. If the group operation is understood as multiplication, we have, for $n\in\mathbb Z$,
$$a^n=\begin{cases}
	\underbrace{a*a*\cdots*a}_{n\text{ times}} & \text{if $n\geq 1$},\\
	e & \text{if $n=0$},\\
	\underbrace{(a^{-1})*(a^{-1})*\cdots*(a^{-1})}_{|n|\text{ times}} & \text{if $n<0$}.
\end{cases}$$

Similar rules apply when the group operation is interpreted as addition, where we denote the inverse of $a$ as $-a$ and write
$$na=\begin{cases}
	\underbrace{a+a+\cdots+a}_{n\text{ times}} & \text{if $n\geq 1$},\\
	e & \text{if $n=0$},\\
	\underbrace{(-a)+(-a)+\cdots+(-a)}_{|n|\text{ times}} & \text{if $n<0$}.
\end{cases}$$

\begin{definition}[element order]
	The order of $a\in G$, denoted $|a|$, is the smallest nonnegative integer $n$ such that $a^n=e$. If there is no such $n$, $|a|=\infty$.
\end{definition}

\begin{example}\phantom{x}
\begin{enumerate}
	\item $\mathbb Z_n=\{0,1,\hdots,n-1\}$ with operation $+$ and identity $0$;
	\item $\text{GL}_n(\mathbb R)=\{\text{all $n\times n$ invertible matrices with entries in $\mathbb R$}\}$ (not abelian);
	\item $D_n=\text{dihedral group of order $2n$}=\text{symmetries of the $n$-gon}$
	\item $S_n=\{\sigma\mid\text{$\sigma$ is a permutation of $\{1,\hdots,n\}$}\}$
\end{enumerate}
\end{example}

\begin{definition}[subgroup]
	A \textbf{subgroup} $H$ of $G$ is a subset $H\subseteq G$ such that $H$ is also a group under the same operation.
\end{definition}

\begin{proposition}[Subgroup Criterion]\label{prop:subgroup_criterion}
	Let $G$ be a group and $H$ a subset of $G$. Then $H$ is a subgroup of $G$ if and only if
	\begin{enumerate}
		\item $e\in H$;
		\item If $a,b\in H$, then $ab\in H$;
		\item If $a\in H$, the $a^{-1}\in H$.
	\end{enumerate}
\end{proposition}

We can collapse the last two conditions into one.

\begin{proposition}[Subgroup Criterion (condensed)]
	Let $G$ be a group and $H$ a subset of $G$. Then $H$ is a subgroup of $G$ if and only if $e\in H$ and $ab^{-1}\in H$ for any $a,b\in H$.
\end{proposition}

\begin{definition}[cyclic subgroup]
	Let $a\in G$. The \textbf{cyclic subgroup} of $G$ generated by $a$, denoted $\langle a\rangle$, is the subset
	$$\langle a\rangle=\{a^n\mid n\in\mathbb Z\}\subseteq G.$$
	We say that $a$ is the \textbf{generator} of $\langle a\rangle$.
\end{definition}

\begin{theorem}
	If $a\in G$, $\langle a\rangle$ of $G$ is a subgroup of $G$.
\end{theorem}

\begin{proof}
	We verify that the criteria in Proposition \ref{prop:subgroup_criterion} hold.
	\begin{enumerate}[label=\textbf{(\alph*)}]
		\item $e\in\langle a\rangle$ since $e=a^0$.
		\item Suppose $x,y\in\langle a\rangle$. Then $x=a^m$ and $y=a^n$ for $m,n\in\mathbb Z$. So $xy=a^ma^n=a^{m+n}\in\langle a\rangle$.
		\item Let $x\in\langle a\rangle$. Then $x=a^n$ for some $n\in\mathbb Z$. We also have $a^{-n}\in\langle a\rangle$ and $xa^{-n}=a^na^{-n}=a^0=e$. So $x^{-1}\in\langle a\rangle$.
	\end{enumerate}
\end{proof}

\begin{definition}[cyclic group]
	A group $G$ is said to be \textbf{cyclic} if $G=\langle a\rangle$ for some $a\in G$.
\end{definition}

\begin{example}
	$\mathbb Z_n=\{0,1,\hdots,n-1\}$ is a cyclic group generated by 1:
	$$\langle 1\rangle=\{1\cdot 0,1\cdot 1,\hdots,1\cdot (n-1)\}=\mathbb Z_n.$$
\end{example}

\begin{theorem}
	$\mathbb Z_n=\langle a\rangle$ if and only if $\gcd(a,n)=1$.
\end{theorem}

\begin{theorem}[Lagrange]
	If $G$ is a finite group and $H$ is a subgroup of $G$, then $|H|$ divides $|G|$.
\end{theorem}

\begin{corollary}
	If $a\in G$, then $|a|$ divides $|G|$.
\end{corollary}

\begin{proof}
	Given $a\in G$, consider the subgroup $\langle a\rangle$. Then $|a|=|\langle a\rangle|$. By Lagrange's Theorem, $|\langle a\rangle|$ divides $|G|$.
\end{proof}

\begin{corollary}
	If $|G|=p$ with $p$ prime, then $G$ is cyclic.
\end{corollary}

\begin{proof}
	Let $a\in G$ be an element other than the identity. So $|a|$ divides $|G|=p$. But $|a|\neq 1$ since $a$ isn't the identity. So we must have $|a|=p$. In particular, $|\langle a\rangle|=p$. Thus, $\langle a\rangle =G$.
\end{proof}

You may notice a thematic question: 

\begin{rmkbox}
	If we know the factorization of $|G|$, what can we say about its structure?
\end{rmkbox}

The next chapter will attempt to tackle this question.

We finish with a sketch of the proof for Lagrange's Theorem.

Fix a subgroup $H$ of $G$. The left coset of $H$ with representative $g$ is the set
$$gH=\{gh\mid h\in H\}.$$
We have the following facts:
\begin{enumerate}
	\item $g_1H=g_2H$ if and only if $g_1^{-1}g_2\in H$;
	\item Either $g_1H=g_2H$ or $g_1H\cap g_2H=\emptyset$, i.e. cosets partition the group;
	\item $|gH|=|H|$.
\end{enumerate}

Suppose $g_1H,g_2H,\hdots,g_nH$ are the distinct left cosets of $H$. Then $G=g_1H\cup g_2H\cup\hdots\cup g_nH$. So $|G|=|g_1H|+|g_2H|+\cdots +|g_n||H|$. But $|g_iH|=|H|$. So $|G|=|H|+\cdots+|H|=n|H|$. So $|H|$ divides $|G|$.