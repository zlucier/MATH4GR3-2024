\section{Lecture 11 — Counting and Burnside's Equation}

We would like to apply group actions to solve counting problems.

\underline{Problem:} We have a flag with six equal stripes. We can colour the bands with red, blue, or green. We want to count the number of possible flags. Below is an example of such a flag.

\begin{center}
\begin{tikzpicture}[yscale=1/12,scale=4]
	\filldraw[MaterialGreen500,draw=black] (0,0) rectangle (1,1);
	\filldraw[MaterialRed500,draw=black] (0,1) rectangle (1,2);
	\filldraw[MaterialRed500,draw=black] (0,2) rectangle (1,3);
	\filldraw[MaterialBlue500,draw=black] (0,3) rectangle (1,4);
	\filldraw[MaterialRed500,draw=black] (0,4) rectangle (1,5);
	\filldraw[MaterialBlue500,draw=black] (0,5) rectangle (1,6);
\end{tikzpicture}
\end{center}

Note that a flag can be represented by a six tuple,
$$(c_1,c_2,c_3,c_4,c_5)$$
with $c_i\in\{\text{\color{MaterialRed500}red},\text{\color{MaterialGreen500}green},\text{\color{MaterialBlue500}blue}\}$.

\begin{center}
\begin{tikzpicture}[yscale=1/12,scale=4]
	\filldraw[MaterialGreen500,draw=black] (0,0) rectangle (1,1);
	\filldraw[MaterialRed500,draw=black] (0,1) rectangle (1,2);
	\filldraw[MaterialRed500,draw=black] (0,2) rectangle (1,3);
	\filldraw[MaterialBlue500,draw=black] (0,3) rectangle (1,4);
	\filldraw[MaterialRed500,draw=black] (0,4) rectangle (1,5);
	\filldraw[MaterialBlue500,draw=black] (0,5) rectangle (1,6);

	\node[right,overlay] at (1,0.5) {$c_1=\text{\color{MaterialGreen500}green}$};
	\node[right,overlay] at (1,1.5) {$c_2=\text{\color{MaterialRed500}red}$};
	\node[right,overlay] at (1,2.5) {$c_3=\text{\color{MaterialRed500}red}$};
	\node[right,overlay] at (1,3.5) {$c_4=\text{\color{MaterialBlue500}blue}$};
	\node[right,overlay] at (1,4.5) {$c_5=\text{\color{MaterialRed500}red}$};
	\node[right,overlay] at (1,5.5) {$c_6=\text{\color{MaterialBlue500}blue}$};
\end{tikzpicture}
\end{center}

Let $X=\{\text{all such six tuples}\}$. We have $|X|=3^6$. Let $\tau$ be the permutation that corresponds to ``flipping'' the flag:
$$\tau=\begin{pmatrix}
	1 & 2 & 3 & 4 & 5 & 6\\
	6 & 5 & 4 & 3 & 2 & 1
\end{pmatrix}.$$
Let $G=\{(1),\tau\}$. Make $X$ into a $G$-set
 by
\begin{align*}
	G\times X &\to X\\
	(\sigma,(c_1,\hdots,c_6)) &\mapsto\begin{cases}
		(c_1,\hdots,c_6) & \text{if $\sigma=(1)$},\\
		(c_6,\hdots,c_1) & \text{if $\sigma=\tau$}.
	\end{cases}
\end{align*}

For any $x\in X$, $\mathcal O_x=\{\sigma\cdot x\mid\sigma\in G\}=\{(c_1,\hdots,c_6),(c_6,\hdots,c_1)\}$, where $x=(c_1,\hdots,c_6)$. So

$$|\mathcal O_x|=\begin{cases}
	1 & \text{if $x=\tau\cdot x$},\\
	2 & \text{if $x\neq\tau\cdot x$}.
\end{cases}$$

Recall that orbits partition $X$:
$$X=\mathcal O_{x_1}\cup\mathcal O_{x_2}\cup\cdots\cup\mathcal O_{x_k}.$$

So the solution to our problem is to count the number of distinct orbits (each orbit consists of distinct flags).

Recall that the stabilizer of $x$ is the set
$$G_x=\{g\in G\mid g\cdot x=x\}.$$

\begin{lemma}\label{lem:burn_lem}
	Suppose $X$ is a $G$-set and $x\sim y$, i.e. $y=g\cdot x$ for some $g\in G$. Then if $G_x\cong G_y$, then $|G_x|=|G_y|$.
\end{lemma}

\begin{proof}
	Let $g\in G$ be such that $y=g\cdot x$ and so $g^{-1}\cdot y=x$. Define a map
	\begin{align*}
		\Phi\colon G_x&\to G_y,\\
		a&\mapsto gag^{-1}
	\end{align*}

	Note $gag^{-1}\in G_y$ since
	\begin{align*}
		gag^{-1}\cdot y&=ga\cdot (g^{-1}\cdot y)\\
		&=ga\cdot x\\
		&=g\cdot (a\cdot x)\\
		&=g\cdot x\\
		&=y
	\end{align*}

	This is a homomorphism since
	\begin{align*}
		\Phi(ab)&=gdbg^{-1}\\
		&=(gag^{-1})(gbg^{-1})\\
		&=\Phi(a)\Phi(b)
	\end{align*}

	It is injective since if
	$$\Phi(a)=gag^{-1}=gbg^{-1}=\Phi(b),$$
	we have $a=b$ by cancellation. It is surjective since if $h\in G_y$, then $g^{-1}hg\in G_x$ since
	\begin{align*}
		(g^{-1}hg)\cdot x&=g^{-1}h\cdot (g\cdot x)\\
		&=g^{-1}h\cdot y\\
		&=g^{-1}\cdot (h\cdot y)\\
		&=g^{-1}\cdot y\\
		&=x.
	\end{align*}
	Thus $\Phi(g^{-1}hg)=g(g^{-1}hg)g^{-1}=h$.
\end{proof}

\begin{theorem}[Burnside]
	Let $G$ be a finite group acting on a set $X$. If $k$ is the number of distinct orbits of $X$, then
	$$k=\frac{1}{|G|}\sum_{g\in G}|X_g|,\quad\text{where }X_g=\{x\mid g\cdot x=x\}.$$
\end{theorem}

\begin{proof}
	We want to count all solutions to $g\cdot x=x$. We can count in two ways.

	\underline{Method 1:} Fix $g$ and count all $x\in X$ such that
	$g\cdot x=x$. So if we sum over all $g\in G$, then number of solutions is
	$$\sum_{g\in G}|X_g|.$$

	\underline{Method 2:} Fix an $x\in X$ and count all $g\in G$ such that $g\cdot x=x$. Summing over all $x$,
	$$\sum_{x\in X}|G_x|.$$

	We will equate these. So
	$$\sum_{g\in G}|X_g|=\sum_{x\in X}|G_x|.$$

	We will first refine the index of summation on the RHS. Recall $X=\mathcal O_{x_1}\cup\cdots\cup\mathcal O_{x_k}$.

	$$\sum_{x\in X}|G_x|=\sum_{x\in\mathcal O_{x_1}}|G_x|+\cdots+\sum_{x\in\mathcal O_{x_k}}|G_x|.$$

	By Lemma \ref{lem:burn_lem}, $|G_x|=|G_y|$ for all $x,y\in\mathcal O_{x_i}$. So
	$$\sum_{x\in\mathcal O_{x_i}}|G_x|=|G_{x_i}||\mathcal O_{x_i}|.$$
	Thus,
	$$\sum_{x\in X}|G_x|=|G_{x_1}||\mathcal O_{x_1}|+\cdots +|G_{x_k}||\mathcal O_{x_k}|.$$

	But $|\mathcal O_{x_i}|=|G|/|G_{x_i}|=[G:G_{x_i}]$. Thus
	$$\sum_{x\in X}|G_x|=|G_{x_1}|\frac{|G|}{|G_{x_1}|}+\cdots +|G_{x_k}|\frac{|G|}{|G_{x_k}|}=k|G|.$$

	So
	$$\sum_{g\in G}|X_g|=k|G|\quad\Leftrightarrow\quad k=\frac{1}{|G|}\sum_{g\in G}|X_g|.$$
\end{proof}

Let's come back around to the flag problem. Our group is $G=\{(1),\tau\}$, so $|G|=2$. We compute:
\begin{align*}
	X_{(1)}&=\{x\mid (1)\cdot x=x\}=X\Rightarrow |X_{(1)}|=3^6\\
	X_\tau&=\{x\mid \tau\cdot x=x\}=\{(c_1,c_2,c_3,c_4,c_5,c_6)\mid c_1=c_6,c_2=c_5,c_3=c_4\}\Rightarrow |X_{\tau}|=3^3
\end{align*}

So the number of flags = the number of orbits which, by Burnside's Theorem, is
$$\frac 12(|X_{(1)}|+|X_\tau|)=378.$$