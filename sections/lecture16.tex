\section{Lecture 16 — Group Theory and Linear Algebra}

Let $\mathcal M_n(\mathbb R)$ be the set of all $n\times n$ matrices with entries in $\mathbb R$. Ignoring matrix multiplication, $\mathcal M_n(\mathbb R)$ is a perfectly valid group under addition (check!). It is however the case that $\mathcal M_n(\mathbb R)$ is \underline{not} a group under matrix multiplication—many matrices (eg. the zero matrix) do not have a defined multiplicative inverse.

Recall that $A\in\mathcal M_n(\mathbb R)$ is invertible if and only if $\det(A)\neq 0$.

\begin{definition}[general linear group]
	Let
	$$\operatorname{GL}_n(\mathbb R)=\{A\in\mathcal M_n(\mathbb R)\mid \det(A)\neq 0\}.$$
	Then $\operatorname{GL}_n(\mathbb R)$ is called the \textbf{general linear group} of degree $n$ over $\mathbb R$.
\end{definition}

We have the following facts:
\begin{itemize}
	\item $\operatorname{GL}_n(\mathbb R)$ is a group under multiplication.
	\item $|{\operatorname{GL}_n(\mathbb R)}|=\infty$.
	\item $\operatorname{GL}_n(\mathbb R)$ is not abelian.
	\item The identity is $I_n$.
	\item Given $A\in \operatorname{GL}_n(\mathbb R)$, its inverse is $A^{-1}$.
\end{itemize}

Recall that $\mathbb R^*=\mathbb R\setminus\{0\}$ is a group under multiplication

\begin{theorem}
	The map $\det\colon\operatorname{GL}_n(\mathbb R)\to\mathbb R^*$ given by $A\mapsto\det(A)$ is a group homomorphism.
\end{theorem}

\begin{proof}
	Let $A,B\in\operatorname{GL}_n(\mathbb R)$. Then $\det(AB)=\det(A)\det(B)$. So $\det$ is a group homomorphism.
\end{proof}

We can use the kernel of $\det$ to find the canonical normal subgroup:
$$\ker(\det)=\{A\in\operatorname{GL}_n(\mathbb R)\mid\det(A)=1\}.$$

\begin{definition}[special linear group]
	The \textbf{special linear group} of degree $n$ over $\mathbb R$, denoted $\operatorname{SL}_n(\mathbb R)$, is the kernel of $\det$. That is, $\operatorname{SL}_n(\mathbb R)$ consists of all $n\times n$ matrices over $\mathbb R$ with determinant 1.
\end{definition}

By the First Isomorphism Theorem,
$$\operatorname{GL}_n(\mathbb R)/\operatorname{SL}_n(\mathbb R)\cong\mathbb R^*.$$

Recall that a matrix $A$ corresponds to a unique linear transformation $T_A\colon\mathbb R^n\to\mathbb R^n$ given by $T_A(\vec x)=A\vec x$.

\begin{example}
	Let $A=\begin{bmatrix}
		1 & 3 \\ 4 & 1
	\end{bmatrix}$. What does $T_A$ do to the unit square?

	Observe the following.

	$$T_A\left(\begin{bmatrix}
		1\\0
	\end{bmatrix}\right)=\begin{bmatrix}
		1 \\4
	\end{bmatrix},\quad T_A\left(\begin{bmatrix}
		0\\1
	\end{bmatrix}\right)=\begin{bmatrix}
		3 \\1
	\end{bmatrix},\quad T_A\left(\begin{bmatrix}
		1\\1
	\end{bmatrix}\right)=\begin{bmatrix}
		4 \\5
	\end{bmatrix}.$$

	\begin{center}
	\begin{tikzpicture}[scale=0.4]
		\footnotesize
		\begin{scope}
		\draw[->] (-0.5,0) -- (5,0);
		\draw[->] (0,-0.5) -- (0,5);
		\fill[main,opacity=0.15] (0,0) -- (1,0) -- (1,1) -- (0,1) -- cycle;
		\draw[thick,main] (0,0) -- (1,0) -- (1,1) -- (0,1) -- cycle;
		\end{scope}
		\draw[->] (6,2.5) -- node[midway,above] {$T_A$} (9,2.5);
		\begin{scope}[shift={(10,0)}]
		\draw[->] (-0.5,0) -- (5,0);
		\draw[->] (0,-0.5) -- (0,5);
		\fill[main,opacity=0.15] (0,0) -- (1,4) -- (4,5) -- (3,1) -- cycle;
		\draw[thick,main] (0,0) -- (1,4) -- (4,5) -- (3,1) -- cycle;
		\end{scope}
	\end{tikzpicture}
	\end{center}
\end{example}

Recall that if $S$ is a set with volume $V$ in $\mathbb R^n$, the volume of $T_A(S)$ is $|{\det(A)}|B$. In particular, if $A\in\operatorname{SL}_n(\mathbb R)$, then the map $T_A\colon\mathbb R^n\to\mathbb R^n$ maps the unit cube to a parallelepiped of volume 1.

Recall that a matrix $A$ is \textbf{orthogonal} if $A^\mathsf T=A^{-1}$.

\begin{example}
	Take $A=\begin{bmatrix}
		\frac 35 & -\frac 45 \\ \frac 45 & \frac 35
	\end{bmatrix}$.

	\begin{align*}
		AA^\mathsf{T}&=\begin{bmatrix}
		\frac 35 & -\frac 45 \\ \frac 45 & \frac 35
	\end{bmatrix}\begin{bmatrix}
		\frac 35 & \frac 45 \\ -\frac 45 & \frac 35
	\end{bmatrix}\\
	&=\begin{bmatrix}
		1 & 0 \\ 0 & 1
	\end{bmatrix}\\
	&=I_2.
	\end{align*}
\end{example}

Equivalently, a matrix $A$ is orthogonal if its columns have norm 1 and are linearly independent.

\begin{definition}
	Let
	$$\mathrm{O}_n(\mathbb R)=\{A\in\operatorname{GL}_n(\mathbb R)\mid A^\mathsf{T}=A^{-1}\}.$$
	Then $\mathrm{O}_n(\mathbb R)$ is called the \textbf{orthogonal group} of degree $n$ over $\mathbb R$.
\end{definition}

It's easy to see that $\mathrm{O}_n(\mathbb R)$ is a group under multiplication (check that it satisfies the subgroup criterion as a subgroup of $\operatorname{GL}_n(\mathbb R)$!).

\begin{proposition}
	If $A\in\mathrm{O}_n(\mathbb R)$, then $\det(A)=\pm 1$.
\end{proposition}

\begin{proof}
	We compute.
	\begin{align*}
		1&=\det(I_n)\\
		&=\det(AA^{-1})\\
		&=\det(A)\det(A^{-1})\\
		&=\det(A)\det(A^\mathsf{T})\\
		&=\det(A)\det(A)\\
		&=\det(A)^2
	\end{align*}
	So $\det(A)=\pm 1$.
\end{proof}

Elements of $\mathrm{O}_n(\mathbb R)$ preserve distance. Recall that, given
$$\vec v=\begin{bmatrix}
	v_1 \\ \vdots \\ v_n
\end{bmatrix}\qquad\text{and}\qquad\vec w=\begin{bmatrix}
	w_1 \\ \vdots \\ w_n
\end{bmatrix},$$
the distance between $\vec v$ and $\vec w$ is
$$d(\vec v,\vec w)=\sqrt{(v_1-w_1)^2+\cdots+(v_n-w_n)^2}=\|\vec v-\vec w\|.$$
This actually characterizes elements of $\mathrm{O}_n(\mathbb R)$.

\begin{theorem}
	$A\in\mathrm{O}_n(\mathbb R)$ if and only if
	$$d(A\vec v,A\vec w)=d(\vec v,\vec w),$$
	for all $\vec v,\vec w\in\mathbb R^n$.
\end{theorem}

Here's the picture, where the dashed lines have equal length.

\begin{center}
\begin{tikzpicture}[scale=0.4]
	\footnotesize
	\begin{scope}
	\draw[->] (-0.5,0) -- (5,0);
	\draw[->] (0,-0.5) -- (0,5);
	\draw[->,very thick,main] (0,0) -- (10:4) node[right] {$\vec v$};
	\draw[->,very thick,main] (0,0) -- (40:5) node[right] {$\vec w$};
	\draw[dashed,main] (10:4) -- (40:5);
	\end{scope}
	\draw[->] (6,2.5) -- node[midway,above] {$T_A$} (9,2.5);
	\begin{scope}[shift={(10,0)}]
	\draw[->] (-0.5,0) -- (5,0);
	\draw[->] (0,-0.5) -- (0,5);
	\draw[->,very thick,main] (0,0) -- (40:4) node[right] {$A\vec v$};
	\draw[->,very thick,main] (0,0) -- (70:5) node[above] {$A\vec w$};
	\draw[dashed,main] (40:4) -- (70:5);
	\end{scope}
\end{tikzpicture}
\end{center}

\begin{definition}
	We define the \textbf{special orthogonal group}, denoted $\operatorname{SO}_n(\mathbb R)$, as
	\begin{align*}
		\operatorname{SO}_n(\mathbb R)&=\operatorname{O}_n(\mathbb R)\cap\operatorname{SL}_n(\mathbb R)\\
		&=\{\text{all $n\times n$ orthogonal matrices with determinant 1}\}.
	\end{align*}
\end{definition}

We consider the special case for $\operatorname{O}_n(\mathbb R)$ when $n=2$.

Note that any $A\in\operatorname{O}_2(\mathbb R)$ is entirely determined by where it takes $\begin{bmatrix}1\\0\end{bmatrix}$ and $\begin{bmatrix}0\\1\end{bmatrix}$.

If $A\begin{bmatrix}1\\0\end{bmatrix}=\begin{bmatrix}a\\b\end{bmatrix}$, then $a^2+b^2=1$. Note that this implies
$$A=\begin{bmatrix}
	a & * \\ b & *
\end{bmatrix}.$$
Since the columns of $A$ must be orthogonal, then either
\begin{align}
	A&=\begin{bmatrix}
	a & b \\ b & -a
\end{bmatrix},\quad\text{or}\label{eq:A1}\\
	A&=\begin{bmatrix}
	a & -b \\ b & a
\end{bmatrix}\label{eq:A2}
\end{align}

If we are in case (\ref{eq:A1}), then
$$A=\begin{bmatrix}
	a & b \\ b & -a
\end{bmatrix}=\begin{bmatrix}
	\cos\theta & \sin\theta \\ \sin\theta & -\cos\theta
\end{bmatrix},\quad\text{for $\theta\in[0,2\pi)$},$$
i.e. $A$ rotates vectors about the origin by an angle $\theta$.

If we are in case (\ref{eq:A2}), then
$$A=\begin{bmatrix}
	a & -b \\ b & a
\end{bmatrix}=\underbrace{\begin{bmatrix}
	a & b \\ b & -a
\end{bmatrix}}_\text{rotation}\underbrace{\begin{bmatrix}
	1 & 0 \\ 0 & -1
\end{bmatrix}}_\text{flip about $x$-axis}.$$