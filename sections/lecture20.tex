\section{Lecture 20 — Review of Rings II}

\begin{definition}[ring homomorphism]
	Let $R$ and $S$ be rings. A \textbf{ring homomorphism} is a function $\varphi\colon R\to S$ such that
	\begin{enumerate}
		\item $\varphi(a+b)=\varphi(a)+\varphi(b)$, and
		\item $\varphi(ab)=\varphi(a)\varphi(b)$.
	\end{enumerate}
\end{definition}

\begin{definition}[ring isomorphism]
	A ring homomorphism $\varphi\colon R\to S$ is called a \textbf{ring isomorphism} if $\varphi$ is a bijection.

	If there exists such a bijection $\varphi$, we say that $R$ and $S$ are \textbf{isomorphic} and write $R\cong S$.
\end{definition}

\begin{proposition}
	Let $\varphi\colon R\to S$ be a ring homomorphism. Then,
	\begin{enumerate}
		\item $\varphi(0_R)=0_S$,
		\item $\varphi(-a)=-\varphi(a)$,
		\item $\varphi(a^n)=\varphi(a)^n$,
		\item $\varphi(R)=\{\varphi(r)\mid r\in R\}\subseteq S$ is a subring, and
		\item If $1_R\in R$ and $1_S\in S$ and if $\varphi$ is surjective, then $\varphi(1_R)=1_S$
	\end{enumerate}
\end{proposition}

\begin{proof}\phantom{x}
	\begin{itemize}
		\item[\textbf{(a)}] Note $0_R=0_R+0_R$. So
		$$\varphi(0_R)=\varphi(0_R+0_R)=\varphi(0_R)+\varphi(0_R).$$
		Subtracting $\varphi(0_R)$ from both sides yields the desired result.
		\item[\textbf{(e)}] To show $\varphi(1_R)=1_S$, we need to show $\varphi(1_R)$ ``acts like'' $1_S$. Take any $b\in S$. Since $\varphi$ is surjective, we have $a\in R$ with $\varphi(a)=b$. So $b=\varphi(a)=\varphi(1_R\cdot a)=\varphi(1_R)\cdot\varphi(a)=\varphi(1_R)\cdot b$. By the same argument, $b=\varphi(a)=\varphi(a\cdot 1_R)=\varphi(a)\cdot\varphi(1_R)=b\cdot\varphi(1_R)$. Since the multiplicative identity is unique, $1_S=\varphi(1_R)$.
	\end{itemize}
\end{proof}

\begin{definition}[kernel]
	The \textbf{kernel} of a ring homomorphism $\varphi\colon R\to S$, denoted $\ker\varphi$, is the set
	$$\ker\varphi=\{r\in R\mid\varphi(r)=0_S\}.$$
\end{definition}

\begin{theorem}
	Let $\varphi\colon R\to S$ be a ring homomorphism. Then,
	\begin{enumerate}
		\item $\ker\varphi$ is an ideal of $R$, and
		\item $\ker\varphi=\{0_R\}$ if and only if $\varphi$ is injective.
	\end{enumerate}
\end{theorem}

\begin{proof}[{\color{main}\sffamily\itshape Proof of (a).}]
	We verify that $\ker\varphi$ satisfies the properties of an ideal.
	\begin{itemize}
		\item $\ker\varphi\neq\emptyset$ since $0_R\in R$ and $\varphi(0_R)=0_S$.
		\item Let $a\in\ker\varphi$ and $r\in R$. Then $\varphi(ra)=\varphi(r)\varphi(a)=\varphi(r)\cdot 0_S=0_S$. So $ra\in\ker\varphi$.
		\item Let $a,b\in\ker\varphi$. Then $\varphi(a-b)=\varphi(a)-\varphi(b)=0_S-0_S=0_S$. So $a-b\in\ker\varphi$.
	\end{itemize}
\end{proof}

A consequence of the above result is that any homomorphism $\varphi\colon R\to S$ gives a quotient ring $R/\ker\varphi$.

\begin{theorem}[First Isomorphism Theorem]
	Let $\varphi\colon R\to S$ be a ring homomorphism. Then
	$$R/\ker\varphi\cong\varphi(R).$$
\end{theorem}

\begin{example}
	Let $R=\mathbb Z$ and $S=\mathbb Z_{2024}$. Define a ring homomorphism
	\begin{align*}
		\varphi\colon\mathbb Z&\to\mathbb Z_{2024},\\
		n&\mapsto n\pmod{2024}.
	\end{align*}
	This map is onto. So $\varphi(\mathbb Z)=\mathbb Z_{2024}$. By the First Isomorphism Theorem, $\mathbb Z/\ker\varphi\cong\mathbb Z_{2024}$. We claim $\ker\varphi=\{2024k\mid k\in\mathbb Z\}$.

	\begin{definition}[principal ideal]
		We call an ideal $I$ of a ring $R$ a \textbf{principal ideal} if there exists $a\in R$ such that
		$$I=\{ra\mid r\in R\}.$$
		We write $I=\langle a\rangle$ and say that $a$ \textbf{generates} the ideal $I$.
	\end{definition}
	So $\ker\varphi=\langle 2024\rangle$. So $\mathbb Z/\langle 2024\rangle\cong\mathbb Z_{2024}$. In fact,
	$$\mathbb Z/\langle m\rangle\cong\mathbb Z_{m}.$$
\end{example}

\begin{theorem}[Second Isomorphism Theorem]
	Let $I$ be a subring of $R$ and $J$ and ideal of $R$. Then $I\cap J$ is an ideal of $U$ and
	$$I/I\cap J\cong (I+J)/J.$$
\end{theorem}

\begin{theorem}[Third Isomorphism Theorem]
	Let $I$ and $J$ be ideals of $R$ with $I\subseteq J\subseteq R$. Then
	$$(R/I)/(J/I)\cong R/J.$$
\end{theorem}

\begin{theorem}[Fourth Isomorphism Theorem]
	Let $I$ be an ideal of $R$. Then there is a one-to-one correspondence between the ideals of $R/I$ and the ideals $J$ of $R$ that contain $I$, i.e.
	$$I\subseteq J\subseteq R.$$
\end{theorem}

Suppose now that $R$ is a commutative ring.

\begin{definition}[maximal ideal]
	An ideal $M$ is a \textbf{maximal ideal} of $R$ if for every ideal $J$ of $R$ with $M\subseteq J\subseteq R$, either $J=M$ or $J=R$.
\end{definition}

\begin{definition}
	An ideal $P$ is a \textbf{prime ideal} of $R$ if $P\neq R$ and whenever $ab\in P$, $a\in P$ or $b\in P$.
\end{definition}

\begin{theorem}
	$M$ is a maximal ideal of $R$ if and only if $R/M$ is a field.
\end{theorem}

\begin{example}
	$\langle 2024\rangle$ in $\mathbb Z$ is \textit{not} maximal since $\mathbb Z/\langle 2024\rangle\cong\mathbb Z_{2024}$ but $\mathbb Z_{2024}$ is not a field.
\end{example}

\begin{example}
	$\langle m\rangle$ in $\mathbb Z$ is a maximal ideal if and only if $m$ is prime.
\end{example}

\begin{theorem}
	$P$ is a prime ideal if and only if $R/P$ is an integral domain.
\end{theorem}

\begin{example}
	$P$ is a prime ideal of $\mathbb Z$ if and only if
	$$P=\langle p\rangle,\text{ ($p$ prime)}\qquad\text{or}\qquad P=\langle 0\rangle.$$
	Note $\mathbb Z/\langle 0\rangle\cong\mathbb Z$ is a domain, so $\langle 0\rangle$ is prime.
\end{example}

\begin{theorem}
	Every maximal ideal is a prime ideal.
\end{theorem}

\begin{proof}
	$M$ is maximal if and only if $R/M$ is a field, which is an integral domain. An ideal $M$ is prime if and only if $R/M$ is an integral domain.
\end{proof}

\begin{remark}
	$\langle 0\rangle$ is prime but not maximal.
\end{remark}