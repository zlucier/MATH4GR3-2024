\section{Lecture 23 — Irreducible Polynomials}

A common theme in previous lectures is that $\mathbb Z$ and $F[x]$ for a field $F$ are ``similar''. In $\mathbb Z$, we have the notion of a prime number. We would like something similar for $F[x]$.

Note that any polynomial $p(x)\in F[x]$ can be factored. Take for example $p(x)=x^2+x+1=\frac1 c(cx^2+cx+c)$ with $c\neq 0$. We need some extra conditions to get away from trivial examples such as this one.

\begin{definition}[reducible/irreducible polynomial]
	A nonconstant polynomial $f(x)\in F[x]$ is \textbf{reducible} if there exist polynomials $g(x),h(x)\in F[x]$ such that $f(x)=g(x)h(x)$ with $0<\deg g(x)<\deg f(x)$ and $0<\deg h(x)<\deg f(x)$. Otherwise, $f(x)$ is \textbf{irreducible}.
\end{definition}

\begin{remark}
	Reducibility depends on the field $F$:
	\begin{itemize}
		\item $x^2-2$ is irreducible in $\mathbb Q[x]$;
		\item $x^2-2=\left(x-\sqrt 2\right)\left(x+\sqrt 2\right)$ is reducible in $\mathbb R[x]$.
	\end{itemize}
\end{remark}

\begin{proposition}
	If $f(x)$ has degree $>1$ and $f(x)$ has a root $\alpha\in F$, then $f(x)$ is reducible.
\end{proposition}

\begin{proof}
	Since $f(\alpha)=0$, $f(x)=(x-\alpha)g(x)$, for some polynomial $g(x)$. But $\deg f(x)>1$. So we must have $\deg g(x)\geq 1$.
\end{proof}

\begin{example}
	Show $p(x)=x^3+x^2+2$ is irreducible in $\mathbb Z[x]$.
	\begin{proof}
		If $p(x)$ was reducible, we would have $p(x)=q(x)r(x)$, where one of $q(x),r(x)$ has degree 1 and the other has degree 2. Say $\deg q(x)=1$. Then $q(x)=ax+b$, $a,b\in\mathbb Z_3$. So $p(x)$ has a root in $\mathbb Z_{3}$. But
		$$p(0)=2,\quad p(1)=4=1,\quad\text{and}\quad p(2)=14=2.$$
		So $p(x)$ has no root, yielding a contradiction. Thus, $p(x)$ is irreducible.
	\end{proof}
\end{example}

\begin{theorem}\phantom{x}
	\begin{enumerate}
		\item $f(x)\in\mathbb C[x]$ is irreducible if and only if $\deg f(x)=1$.
		\item $f(x)\in\mathbb R[x]$ is irreducible if and only if $\deg f(x)=1$ or $f(x)=ax^2+bx+c$ with $b^2-4ac<0$.
	\end{enumerate}
\end{theorem}

We can reduce polynomials over $\mathbb Q$ to polynomials over $\mathbb Z$ up to a rational factor.

\begin{lemma}
	Let $p(x)\in\mathbb Q[x]$. Then there exist $r,s,a_0,a_1,\hdots,a_n\in\mathbb Z$ such that $\gcd(r,s)=1$, $\gcd(a_0,a_1,\hdots,a_n)=1$ and
	$$p(x)=\frac rs(a_nx^n+\cdots+a_1x+a_0).$$
\end{lemma}

\begin{example}
	Take $p(x)=\frac{3}{5}+\frac 23x+\frac{3}{10}x^2$.

	\begin{align*}
		p(x)&=\frac{3}{5}+\frac 23x+\frac{3}{10}x^2\\
		&=\frac{1}{5\cdot 3\cdot 10}\left(3\cdot 10\cdot 3+5\cdot 10\cdot 2x+5\cdot 3\cdot 3x^2\right)\\
		&=\frac{5}{5\cdot 3\cdot 10}\left(3\cdot 5\cdot 3+10\cdot 2x+3\cdot 3x^2\right)\\
		&=\frac{1}{30}(18+20x+9x^2)
	\end{align*}
\end{example}

\begin{lemma}[Gauss]\label{lem:gauss}
	Let $p(x)$ be a polynomial in $\mathbb Z[x]$ such that $p(x)=\alpha(x)\beta(x)$ with $\alpha(x),\beta(x)\in\mathbb Q[x]$. Then $p(x)=a(x)b(x)$ with $a(x),b(x)\in\mathbb Z[x]$ and $\deg a(x)=\deg\alpha(x)$ and $\deg b(x)=\deg\beta(x)$.
\end{lemma}

\begin{remark}
	Judson states that $p(x)$ needs to be monic but this is not the case.
\end{remark}

\begin{corollary}
	Let $p(x)=x^n+a_{n-1}x^{n-1}+\cdots+a_1x+a_0$ with $a_i\in\mathbb Z$ and $a_0\neq 0$. if $p(x)$ has a root $\frac rs\in \mathbb Q$, then it has a root $\alpha\in\mathbb Z$ and $\alpha$ divides $a_0$.
\end{corollary}

\begin{proof}
	Suppose $p(r/s)=0$. So $p(x)=(x-r/s)q(x)$ and $(x-r/s),q(x)\in\mathbb Q[x]$. By Gauss' Lemma (\ref{lem:gauss}), there exist $(x-\alpha),\tilde q(x)\in\mathbb Z$ such that $p(x)=(x-\alpha)\tilde q(x)$. Thus $\alpha$ is a root of $p(x)$ and $\alpha\in\mathbb Z$.

	If we write
	$$\tilde q(x)=b_{n-1}x^{n-1}+\cdots+b_1x+b_0,$$
	we have
	\begin{align*}
		p(x)&=x^n+a_{n-1}x^{n-1}+\cdots+a_1x+a_0\\
		&=(x-\alpha)(b_{n-1}x^{n-1}+\cdots+b_1x+b_0)\\
		&=\cdots +\alpha b_0.
	\end{align*}

	So $a_0=\alpha b_0$ and thus $\alpha$ divides $a_0$.
\end{proof}

\begin{example}
	Show $p(x)=x^3-7x+5$ has no roots in $\mathbb Q$.

	\begin{proof}
		If $p(x)$ did have a root, it would have an integer root $\alpha$, we would have that $\alpha$ divides 5. So $\alpha=\pm 1,\pm 5$. But $p(\alpha)\neq 0$ for these choices of $\alpha$. So $p(x)$ has no roots in $\mathbb Q$.
	\end{proof}
\end{example}

\begin{theorem}[Eisenstein's Criterion]\label{thm:eisenstein}
	Let $p(x)=a_nx^n+a_{n-1}x^{n-1}+\cdots+a_1x+a_0\in\mathbb Z[x]$. Suppose there exists a prime $p$ such that
	\begin{enumerate}
		\item $p$ divides $a_0,a_1,\hdots,a_{n-1}$,
		\item $p$ does not divide $a_n$, and
		\item $p^2$ does not divide $a_0$.
	\end{enumerate}
	Then $p(x)$ is irreducible.
\end{theorem}

\begin{proof}
	The proof uses Gauss' Lemma (\ref{lem:gauss}). If $p(x)$ was reducible, we would have
	$$p(x)=(b_rx^r+\cdots +b_0)(c_sx^s+\cdots+c_0)\in\mathbb Z[x],$$
	where $r+s=n$. Since $p^2$ does not divide $a_0=b_0c_0$, but $p$ divides $a_0$ then $p$ does not divide one of $b_0,c_0$. Say $p$ does not divide $b_0$ but divides $c_0$. Since $a_n=b_rc_s$ and $p$ does not divide $a_n$, $p$ divides neither $b_r$ nor $c_s$. Let $k$ be the smallest integer such that $p$ does not divide $c_k$. Then
	$$a_k=b_0c_k+b_1c_{k-1}+\cdots b_kc_0\Leftrightarrow b_0c_k=a_k-b_1c_{k-1}-\cdots -b_kc_0.$$
	If $k<n$, then $p$ divides the RHS but not the left. So we must have $k=n$. But now this implies $\deg(c_sx^s+\cdots +c_0)\geq n$, yielding a contradiction. Thus, $p(x)$ is irreducible.
\end{proof}

\begin{example}
	$x^n-2024$ is irreducible over $\mathbb Q[x]$ for all $n\geq 2$. Consider $p=23$. Then $p$ divides 2024 but $p^2$ does not. Obviously $p$ does not divide 1. So by Eisenstein's criterion, $x^n-2024$ is irreducible.
\end{example}