\section{Lecture 24 — Ideals in \textit{F}[\textit{x}]}

Recall that an ideal $I$ in a commutative ring $R$ is \textbf{principal} if there exists an element $a\in R$ such that
$$I=\{ra\mid r\in R\}=\langle a\rangle.$$

Recall also that every ideal of $\mathbb Z$ is principal. Extending from the similarities between $\mathbb Z$ and $F[x]$ seen in previous lectures, it is also the case that every ideal of $F[x]$ is principal.

\begin{theorem}\label{thm:principal_ideal_Fx}
	Every ideal $I$ of $F[x]$ is principal.
\end{theorem}

\begin{proof}
	If $I=\{0\}$, then $I=\langle 0\rangle$ and clearly $I$ is principal. So assume $I\neq\{0\}$. Let $p(x)\in I$ with $p(x)\neq 0$ and $\deg p(x)\leq\deg q(x)$ for all $q(x)\in I$. If $\deg p(x)=0$, then $p(x)=c$ for some $c\in F$. So $c\in I$. But $c^{-1}\in F\subset F[x]$, so $c^{-1}c=1\in I$. Thus, $I=F[x]=\langle 1\rangle$. So suppose $\deg p(x)>0$. We claim $I=\langle p(x)\rangle$.

	Since $p(x)\in I$, clearly $\langle p(x)\rangle\subseteq I$.

	Let $t(x)\in I$. By the Division Algorithm,
	$$t(x)=p(x)q(x)+r(x),$$
	with $r(x)=0$ or $\deg r(x)<\deg p(x)$. If $r(x)\neq 0$, we have
	$$r(x)=\underbrace{t(x)}_{\in I}-\underbrace{p(x)}_{\in I}q(x)\in I.$$
	But $\deg r(x)<\deg p(x)$, and $p(x)$ should have the smallest degree in $I$. This is a contradiction. So it must be that
	$$t(x)=p(x)q(x)\in\langle p(x)\rangle.$$
\end{proof}

\begin{example}
	Theorem \ref{thm:principal_ideal_Fx} is not true in general for $F[x,y]$. Consider $\langle x^2,y\rangle=\{f(x,y)x^2+g(x,y)y\mid f(x,y),g(x,y)\in F[x,y]\}$. We claim $\langle x^2,y\rangle$ is not principal.

	Suppose $\langle x^y,y\rangle=\langle p(x,y)\rangle$ for some $p(x,y)\in F[x,y]$. So $p(x,y)$ divides both $x^2$ and $y$. Thus, $p(x,y)$ must be constant, i.e. $p(x,y)=c$ for some $c\in F$. Then $\langle x^2,y\rangle=\langle c\rangle=\langle 1\rangle$. But $1\not\in\langle x^2,y\rangle$; a contradiction.
\end{example}

Recall that in $\mathbb Z$, $\langle a\rangle$ is a maximal ideal if and only if $a$ is prime. We provide a similar result for $F[x]$.

\begin{theorem}
	In $F[x]$, $\langle f(x)\rangle$ is a maximal ideal if and only if $f(x)$ is irreducible.
\end{theorem}

\begin{proof}
	First, suppose $I=\langle f(x)\rangle$ is maximal and $f(x)=p(x)q(x)$ for some $p(x),q(x)\in F[x]$. So $f(x)\in\langle p(x)\rangle$. Thus, $I\subseteq\langle p(x)\rangle$. Because $I$ is maximal, either $I=\langle p(x)\rangle$ or $\langle p(x)\rangle= F[x]$. In the case that $I=\langle p(x)\rangle$, $p(x)\in\langle f(x)\rangle$. So $\deg f(x)\leq\deg p(x)\leq\deg f(x)$. Then it must be that $f(x)=p(x)$. If $\langle p(x)\rangle =F[x]$, then $1\in\langle p(x)\rangle$. So $p(x)$ divides 1 and thus $\deg p(x)=0$. So $f(x)$ is irreducible.

	Now let $I=\langle f(x)\rangle$ with $f(x)$ irreducible and suppose $J$ is a principal ideal such that $I\subseteq J\subseteq F[x]$. Since $J=\langle q(x)\rangle$ for some $q(x)\in F[x]$, we have $f(x)\in\langle q(x)\rangle$, so $f(x)=p(x)q(x)$ for some $p(x)\in F[x]$. Since $f(x)$ is irreducible, either $\deg q(x)=0$ or $\deg q(x)=\deg f(x)$. In the case that $\deg q(x)=0$, $q(x)=c$ for some $c\in F$. So $J=F[x]$. In the case that $\deg q(x)=\deg f(x)$, $f(x)=cq(x)$ and thus $\langle f(x)\rangle=\langle q(x)\rangle$. So $\langle f(x)\rangle$ is maximal.
\end{proof}

\subsection*{Practice Problems}

\begin{enumerate}[label={\sffamily\bfseries\color{main}\Alph*.}]
	\item Apply the division algorithm to $a(x)=4x^5-x^3+x^2+4$ and $b(x)=x^3-2$ in $\mathbb Z_5[x]$.
	\item For any polynomial $p(x)\in\mathbb R[x]$, we know $p(x)$ has at most $\deg p(x)$ roots. Show that is false in $\mathbb Z_{10}[x]$.
	\item Prove the rational root test: Suppose
	$$p(x)=a_nx^n+\cdots +a_1x+a_0\in\mathbb Z[x].$$
	If $\frac rs\in\mathbb Q$ with $\gcd(r,s)=1$, then $r$ divides $a_0$ and $s$ divides $a_n$. Use this to show $7x^2+2$ has no real roots.
	\item Prove $x^p+a$ is reducible in $\mathbb Z_p[x]$ for any $a\in\mathbb Z_p$ where $p$ is prime.
\end{enumerate}