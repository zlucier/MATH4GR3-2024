\section{Lecture 27 — Principal Ideal Domains}

Recall that an integral domain is a principal ideal domain (PID) if every ideal in the domain is principal. Common examples are $\mathbb Z$ and $F[x]$ for $F$ a field. The goal of this lecture is to show that all PIDs are UFDs.

\begin{lemma}
	Let $D$ be a PID and let $I_1,I_2,\hdots$ be a collection of ideals of $D$ such that
	$$I_1\subseteq I_2\subseteq\cdots.$$
	Then there is an integer $N$ such that
	$$I_N=I_{N+1}=I_{N+2}=\cdots.$$
\end{lemma}

\begin{proof}
	Let $I=\bigcup_{i=1}^\infty I_i$. We claim that $I$ is an ideal. Indeed,
	\begin{itemize}
		\item $I\neq\emptyset$ since $0\in I_1\subseteq I$.
		\item Let $a,b\in I$. Then $a\in I_i$ and $b\in I_j$ for some $i,j$. Without loss of generality, suppose $i\leq j$. Then $a\in I_i\subseteq I_j$, so $a,b\in I_j$. Thus, $a-b\in I_j\subseteq I$.
		\item Let $a\in I$. So $a\in I_i$ for some $i$. For any $r\in D$, $ra\in I_i\subseteq I$.
	\end{itemize}
	Since $D$ is a PID, there exists an element $d\in D$ such that $I=\langle d\rangle$. Since $d\in I=\bigcup_{i=1}^\infty I_i$, there exists and integer $N$ such that $d\in I_N$. So
	$$\langle d\rangle\subseteq I_N\subseteq I_{N+1}\subseteq\cdots\subseteq I=\langle d\rangle.$$
	Therefore, $\langle d\rangle=I_N=I_{N+1}=\cdots$.
\end{proof}

\begin{definition}[Noetherian ring]
	A ring $R$ is a \textbf{Noetherian ring} if it satisfies the ascending chain condition, i.e. for any chain of ideals $I_1\subseteq I_2\subseteq\cdots$, there is an integer $N$ such that $I_N=I_{N+1}=\cdots$.
\end{definition}

\begin{corollary}
	Every PID is Noetherian.
\end{corollary}

\begin{lemma}\label{lem:max_ideal}
	Let $S$ be a nonempty collection of ideals in a PID. Then there is an ideal $J\in S$ such that for all $I\in S$ with $J\subseteq I$, $J=I$, i.e. $S$ has a maximal element.
\end{lemma}

\begin{proof}
	Suppose for a contradiction that $S$ had no such maximal element. Take $I_1\in S$. Since $I_1$ is not maximal, there exists $I_2\in S$ such that $I_1\subsetneq I_2$. Again, $I_2$ is not maximal so there is $I_3\in S$ such that $I_1\subsetneq I_2\subsetneq I_3$. Continuing in this way,
	$$I_1\subsetneq I_2\subsetneq I_3\subsetneq\cdots.$$
	But this contradicts the ascending chain condition and the fact that every PID is Noetherian.
\end{proof}

\begin{lemma}\label{lem:pid_irred_prod}
	Let $R$ be a PID. If $a\in R$ is not a unit, then $a$ can be written as a product of irreducible elements.
\end{lemma}

\begin{proof}
	Let $S$ be the collection of principal ideals $\langle a\rangle$ of $R$ such that $a$ is not a unit and $a$ cannot be written as a product of irreducibles. Our goal is to show $S=\emptyset$.

	Suppose otherwise for a contradiction, i.e. $S\neq\emptyset$. Then by Lemma \ref{lem:max_ideal}, there exists a maximal $\langle a\rangle\in S$. But we also know that $a=bc$ with $a$ not reducible, so $b$ and $c$ are not units. But then
	$$\langle a\rangle\subseteq\langle b\rangle\quad\text{and}\quad\langle a\rangle\subseteq\langle c\rangle.$$
	So $\langle b\rangle,\langle c\rangle\notin S$. So $b$ and $c$ can be factored into irreducibles as $b=p_1\cdots p_r$ and $c=q_1\cdots q_s$. But then $a=bc=p_1\cdots p_rq_1\cdots q_s$ is a product of irreducibles, yielding a contradiction. So $S=\emptyset$.
\end{proof}

\begin{theorem}\label{thm:pid_is_ufd}
	Every PID is also a UFD.
\end{theorem}

\begin{proof}
	Let $D$ be a PID. Let $a\in D$ with $a$ not a unit. By Lemma \ref{lem:pid_irred_prod}, $a$ can be written as a product of irreducibles.

	Suppose $a=p_1\cdots p_r$ and $a=q_1\cdots q_s$ are two ways to write $a$ as a product of irreducibles. Without loss of generality, suppose $r\leq s$. Since $D$ is a PID, $p_1$ is also prime. Since $p_1|q_1\cdots q_s$, we must have $p_1|q_i$ for some $i$. Relabeling, assume $p_1|q_1$, that is, $q_1=u_1p_1$. Note that $u_1$ must be a unit since $q_1$ is irreducible. Thus,
	$$p_1p_2\cdots p_r=u_1p_1q_2\cdots q_s.$$
	In the case that $r<s$, we end up with
	$$1=u_1\cdots u_rq_{r+1}\cdots q_s.$$
	But this cannot happen, since the $q_i$'s are not units. So it must be that $r=s$ and thus $p_i=u_iq_i$ for all $i$. As such, the factorization for $a$ is unique.
\end{proof}

\begin{corollary}
	If $F$ is a field, $F[x]$ is a UFD.
\end{corollary}

Note that the converse of Theorem \ref{thm:pid_is_ufd} is false; there are UFDs that are not PIDs.

\begin{example}
	$F[x_1,\hdots,x_n]$ is a UFD but not a PID.
\end{example}

\begin{definition}[Euclidean domain, Euclidean valuation]
	Let $D$ be an integral domain. Suppose that there is a function $v\colon D\setminus\{0\}\to\mathbb N$ that satisfies the following:
	\begin{enumerate}
		\item If $a,b\in D$, then $v(a)\leq v(ab)$.
		\item Let $a,b\in D$ with $b\neq 0$. Then there exist $q,r\in d$ such that $a=bq+r$ with $r=0$ or $v(r)<v(b)$.
	\end{enumerate}
	Then $D$ is called a \textbf{Euclidean domain} and $v$ is called a \textbf{Euclidean valuation}.
\end{definition}

The function $v$ associates to the elements of $D$ a ``size''.

\begin{example}
	Take $D=\mathbb Z$. We may use
	\begin{align}
		v\colon\mathbb Z\setminus\{0\}&\to\mathbb N,\\
		a&\mapsto |a|.
	\end{align}

	Now take $D=F[x]$. We may use
	\begin{align}
		v\colon\mathbb F[x]\setminus\{0\}&\to\mathbb N,\\
		p(x)&\mapsto \deg p(x).
	\end{align}

	(Check that the above functions satisfy the conditions of a valuation!)

	So $\mathbb Z$ and $F[x]$ are Euclidean domains.
\end{example}

Our goal for next lecture is to show that every Euclidean domain is a PID.