\section{Lecture 28 — Euclidean Domains and Factoring in \textit{D}[\textit{x}]}

A \textbf{Euclidean Domain} is a domain in which there exists a division algorithm

\begin{definition}
	A domain $D$ is a \textbf{Euclidean Domain} if there is a valuation $v\colon D\setminus\{0\}\to\mathbb N$ such that
	\begin{enumerate}
		\item $v(a)\leq v(ab)$ for all $a,b\in D\setminus\{0\}$, and
		\item for all $a,b\in D$ with $b\neq 0$, there exist $q,r\in D$ such that $a=bq+r$ with $r=0$ or $v(r)<v(b)$ 
	\end{enumerate}
\end{definition}

\begin{example}
	If $D=\mathbb Z$, we use the valuation $v\colon \mathbb Z\setminus\{0\}\to\mathbb N$ defined by $a\mapsto |a|$.

	If $D=F[x]$, we use the valuation $v\colon F[x]\setminus\{0\}\to\mathbb N$ defined by $p(x)\mapsto\deg p(x)$.
\end{example}

\begin{example}
	$\mathbb Z[i]=\{a+bi\mid a,b\in\mathbb Z\}$, $i^2=-1$. This is a ring with the usual multiplication and addition. Define
	\begin{align*}
		v\colon \mathbb Z[i]\setminus\{0\}&\to\mathbb N,\\
		a+bi&\mapsto a^2+b^2.
	\end{align*}

	We claim that $v$ is a valuation.
	\begin{proof}
		We check the properties. Let $x=a+bi$ and $y=c+di$. Then
		\begin{align*}
			xy&=(a+bi)(c+di)=(ac-bd)+(ad+bc)i
		\end{align*}
		So
		\begin{align*}
			v(xy)&=(ac-bd)^2+(ad+bc)^2=(ac)^2+(bd)^2+(ad)^2+(bc)^2
		\end{align*}
		Note $v(x)=v(x)v(y)$ (check!). This is not true in general. But then $v(x)\leq v(x)v(y)=v(xy)$ is true.

		Now let $z=a+bi$ and $w=c+di$ with $w\neq 0$. Viewed as elements of $\mathbb Q[i]=\{p+qi\mid p,q\in\mathbb Q\}$,
		\begin{align*}
			\frac{z}{w}&=\frac{a+bi}{c+di}\\
			&=\frac{(a+bi)(c-di)}{(c+di)(c-di)}\\
			&=\frac{(ac+bd)+(bc-ad)i}{c^2+d^2}
		\end{align*}

		Now write
		$$\frac{ac+bd}{c^2+d^2}=m_1+\frac{n_1}{c^2+d^2},$$
		where $\left|\frac{n_1}{c^2+d^2}\right|\leq\frac 12$ and $m_1$ is the closest integer to $\frac{ac+bd}{c^2+d^2}$. Also,
		$$\frac{bc-ad}{c^2+d^2}=m_2+\frac{n_2}{c^2},$$
		where $\left|\frac{n_2}{c^2+d^2}\right|\leq\frac 12$ and $m_2$ is the closest integer to $\frac{bc-ad}{c^2+d^2}$. So
		$$\frac{z}{w}=(m_1+m_2i)+\frac{n_1+n_2i}{c^2+d^2}.$$
		Now
		\begin{align*}
			z&=\frac{z}{w}\cdot w\\
			&=(m_1+m_2i)(c+d_i)+\frac{n_1+n_2i}{c^2+d^2}(c+di)\\
			&=wq+r.
		\end{align*}
		Note $z,w,q\in\mathbb Z[i]$, so $z-wq=r\in\mathbb Z[i]$. Then
		\begin{align*}
			v(r)&=v\left(\frac{n_1+n_2i}{c^2+d^2}(c+di)\right)\\
			&=v(c+di)v\left(\frac{n_1+n_2i}{c^2+d^2}\right)\\
			&=v(c+di)\left(\left({n_1}{c^2+d^2}\right)^2+\left({n_2}{c^2+d^2}\right)^2\right)\\
			&\leq v(c+di)(\frac 14+\frac 14)\\
			&=\frac12\leq v(c+di)
			&<v(c+di).
		\end{align*}
	\end{proof}
\end{example}

\begin{theorem}
	Every ED is a PID.
\end{theorem}

\begin{proof}
	Let $D$ be a Euclidean Domain. Let $I\subset D$ be an ideal. If $I=\{0\}$, then $I=\langle 0\rangle$. So suppose $I\neq\{0\}$. Let $a\in I$ with $v(a)\leq v(b)$ for all $b\in I$ different from $a$. We claim $I=\langle a\rangle$.

	Since $a\in I$, $\langle a\rangle\subseteq I$. Now let $b\in I$. By the division algorithm, $b=aq+r$ with $r=0$ or $v(r)<v(a)$. If $r\neq 0$, then $r=b-aq\in I$ and then $v(r)<v(a)$ is a contradiction to the choice of $a$. So $r=0$ and thus $I=\langle a\rangle$.
\end{proof}

\begin{corollary}
	Every ED is a UFD.
\end{corollary}

\begin{remark}
	Proving a domain is \underline{not} a Euclidean Domain is difficult. It implies showing no such valuation exists.
\end{remark}

\begin{example}
	Let
	$$D=\mathbb Z\left[\frac{1+\sqrt{-19}}{2}\right]=\left.\left\{a+b\left(\frac{1+\sqrt{-19}}{2}\right)~\right|~ a,b\in\mathbb Z\right\}.$$
	To sketch out the proof, suppose there was a valuation $v\colon D\setminus\{0\}\to\mathbb N$. We need to check that the only units in the ring are $\pm 1$. So take any $a\in D\setminus\{0,1,-1\}$ with $v(a)$ as small as possible. For any $b\in D$, we have $b=aq+r$ with $r=0$ or $v(r)<v(a)$. But for units $v(-1)=v(1)<v(a)$. So the only choices for $r$ are $0,1,-1$. So $D/\langle a\rangle\cong\mathbb Z_2$ or $D/\langle a\rangle\cong\mathbb Z_3$.

	In $D$, $x^2+x+5$ has roots $\alpha=\frac{-1\pm\sqrt{-19}}{2}$. But $x^2+x+5$ has no roots in $\mathbb Z_2$ or $\mathbb Z_3$ so $D/\langle a\rangle\not\cong\mathbb Z_2,\mathbb Z_3$.
\end{example}

\begin{center}
	\begin{tikzpicture}[draw=main,thick]
		\footnotesize\sffamily
		\node (A) at (0,0) {Rings};
		\node[align=center] (B) at (-1,1) {Commutative\\Rings};
		\node[align=center] (C) at (1,1) {Rings with\\identity};
		\node[align=center] (D) at (-1,2) {Integral\\domains};
		\node[align=center] (E) at (-1,3) {UFDs};
		\node[align=center] (F) at (-1,4) {PIDs};
		\node[align=center] (G) at (-1,5) {EDs};
		\node[align=center] (H) at (1,5) {Division\\ Rings};
		\node[align=center] (I) at (0,6) {Fields};
		\draw (A) edge (B) edge (C);
		\draw (C) edge (D) edge (H);
		\draw (H) edge (I);
		\draw (B) -- (D) -- (E) -- (F) -- (G) -- (I);
	\end{tikzpicture}
\end{center}

\begin{theorem}
	If $D$ is a UFD, then $D[x]$ is also a UFD.
\end{theorem}

\begin{corollary}
	If $D$ is a UFD, then $D[x_1,\hdots,x_n]$ is a UFD.
\end{corollary}

A special case with applications in algebraic geometry is $\mathbb C[x_1,\hdots,x_n]$.