\section{Lecture 29 — Field Extensions}

The goal for this chapter is to further investigate fields (domains in which each element has an inverse). Standard examples of fields are $\mathbb Q,~\mathbb R,~\mathbb C$ and $\mathbb Z_p$.

A guiding question is as follows: \textit{If $F$ is a field and $p(x)\in F[x]$, is there a ``bigger'' field in which $p(x)$ has a root?}

\begin{example}
	Consider $x^2+1\in\mathbb R[x]$. This polynomial has no roots in $\mathbb R$ but if we make $\mathbb R$ ``bigger'' and consider the complex numbers $\mathbb C$, of which $\mathbb R$ is a subfield, then $x^2+1$ has the root $i$.
\end{example}

\begin{definition}
	A field $E$ is an \textbf{extension} of $F$ if $F$ is a subfield of $E$. We call $F$ the \textbf{base field} of the extension.
\end{definition}

\begin{remark}
	Sometimes we may relax the above definition up to isomorphism. That is, we may say $E$ is an extension of $F$ if $E$ has a subfield $F'$ that is isomorphic to $F$. For example, say the field of fractions $F_{\mathbb Z}$ and $\mathbb Q$ are extensions of $\mathbb Z$.
\end{remark}

\begin{example}\phantom{x}
	\begin{itemize}
		\item $\mathbb C$ is an extension of $\mathbb R$.
		\item $\mathbb R$ is an extension of $\mathbb Q$.
	\end{itemize}
\end{example}

Recall that if $p(x)\in F[x]$ is an irreducible polynomial, then
$$F[x]/\langle p(x)\rangle$$
is a field (equivalently if $\langle p(x)\rangle$ is a maximal ideal). This gives us a way to construct fields.

\begin{example}
	Consider $x^2-2\in\mathbb Q[x]$. Then $E=Q[x]/\langle x^2-2\rangle$ is a field. Note that $E$ contains a ``copy'' of $\mathbb Q$.
	$$\mathbb Q\cong\{a+\langle x^2-2\rangle\mid a\in\mathbb Q\}\subseteq E.$$
	So $E$ is an extension of $\mathbb Q$.

	Observe that if $p(x)=a_0+a_1x+\cdots+a_nx^n\in F[x]$, we can view it as
	$$p(x)=a_0x^0+a_1x^1+\cdots+a_nx^n\in F[x].$$
	Now let
	$$\alpha=x+\langle x^2-2\rangle\in E.$$
	Then
	\begin{align*}
		\alpha^2+2\alpha^0&=(x+\langle x^2-2\rangle)^2-2(x+\langle x^2-2\rangle)^0\\
		&=(x^2+\langle x^2-2\rangle)-2(1+\langle x^2-2\rangle)\\
		&=(x^2-2\cdot 1)+\langle x^2-2\rangle\\
		&=0+\langle x^2-2\rangle.
	\end{align*}
	So $\alpha$ is a root of $x^2-2$.
\end{example}

\begin{theorem}[Fundamental Theorem of Field Theory]
	Let $F$ be a field and $p(x)\in F[x]$ be nonconstant. Then there exists an extension $E$ of $F$ such that $p(x)$ has a root $\alpha\in E$.
\end{theorem}

\begin{proof}
	Since $F[x]$ is a PID, it is also a UFD. So we can uniquely factor $p(x)$:
	$$p(x)=p_1(x)p_2(x)\cdots p_r(x),$$
	where $p_i(x)$ is irreducible for each $i$. So it is enough to prove the result for $p_1$ since it is irreducible. We have that
	$$E=F[x]/\langle p(x)\rangle$$
	is a field. We claim that $E$ is an extension of $F$ (or has a subfield isomorphic to $F$). To give a sketch, define
	\begin{align*}
		\varphi\colon F&\to E,\\
		a&\mapsto a+\langle p(x)\rangle.
	\end{align*}

	One can check that $\varphi$ is indeed a ring homomorphism. Observe that $\varphi$ is injective since
	$$\varphi(a)=\varphi(b)\quad\Leftrightarrow\quad a+\langle p(x)\rangle=b+\langle p(x)\rangle\quad\Leftrightarrow\quad a-b\in\langle p(x)\rangle,$$
	but $\deg p(x)\geq 1$ and $\deg (a-b)=0$. So it must be that $a=b$. Thus, $F\cong\im\varphi\subseteq E$.

	Let $\alpha=x+\langle p(x)\rangle$ be the class of $x$ in $E$. If $p_1=a_nx^n+\cdots+a_1x^1+a_0x^0$, then
	\begin{align*}
		p_1(\alpha)&=a_n\alpha^n+\cdots+a_1\alpha^1+a_0\alpha^0\\
		&=a_n(x^n+\langle p(x)\rangle)+\cdots+a_0(x^0+\langle p(x)\rangle)\\
		&=a_nx^n+\cdots +a_0+\langle p(x)\rangle\\
		&=p(x)+\langle p(x)\rangle\\
		&=\langle p(x)\rangle.
	\end{align*}
	So $\alpha$ is a root.
\end{proof}

\begin{proposition}
	If $p(x)$ is irreducible, there exists a bijection between the elements of $F[x]/\langle p(x)\rangle$ and those of the set
	$$R=\{r(x)\mid r(x)\in F[x]\text{ and }\deg r(x)<\deg p(x)\}.$$
\end{proposition}

\begin{proof}
	Let $g(x)+\langle p(x)\rangle\in F[x]/\langle p(x)\rangle$. By the division algorithm, there exist $q_g(x),r_g(x)\in F[x]$ such that
	$$g(x)=p(x)q_g(x)+r_g(x).$$
	Therefore,
	$$g(x)-r_g(x)=p(x)q_g(x)\in\langle p(x)\rangle,$$
	and in particular,
	$$g(x)+\langle p(x)\rangle=r_g(x)+\langle p(x)\rangle.$$
	Define
	\begin{align*}
		\varphi\colon F[x]/\langle p(x)\rangle &\to R,\\
		g(x)+\langle p(x)\rangle &\mapsto r_g(x).
	\end{align*}
	One can then show that $\varphi$ is indeed a bijection.
\end{proof}

\begin{example}
	What are the irreducible elements of $F=\mathbb Z_2[x]/\langle x^2+x+1\rangle$?
	\begin{solution}
		The polynomial $p(x)=x^2+x+1$ is irreducible, since neither 0 nor 1 are roots of $p(x)$. So the elements of $F$ are in one-to-one correspondence with polynomials of degree less than 2.
		$$F=\left\{\begin{array}{r}
		0+\langle x^2+x+1\rangle,\\1+\langle x^2+x+1\rangle,\\x+\langle x^2+x+1\rangle,\\x^2+\langle x^2+x+1\rangle\phantom{,}\end{array}\right\}$$
	\end{solution}
\end{example}