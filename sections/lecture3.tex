\section{Lecture 3 — The Fundamental Theorem of Finite Abelian Groups I: Setting the stage}

We'll begin with a motivating question: How many ``distinct'' groups are there of order $n\geq 1$?

What do we mean by ``distinct''? Take for example $U(8)=\{a\mid\gcd(a,8)=1,\,a\in\{0,\hdots,7\}\}=\{1,3,5,7\}$, which is a group under multiplication. Consider also $\mathbb Z_2\times\mathbb Z_2=\{(0,0),(0,1),(1,0),(1,1)\}$ under addition. These groups are \underline{not} distinct. This can be seen by constructing Cayley tables.

\begin{center}
\begin{tabular}{c | c c c c c}
	$U(8)$ & 1 & 3 & 5 & 7\\ \hline
	1 & 1 & 3 & 5 & 7\\
	3 & 3 & 1 & 7 & 5\\
	5 & 5 & 7 & 1 & 3\\
	7 & 7 & 5 & 3 & 1
\end{tabular}\qquad\begin{tabular}{c | c c c c c}
	$Z_2\times\mathbb Z_2$ & $(0,0)$ & $(1,0)$ & $(0,1)$ & $(1,1)$\\ \hline
	$(0,0)$ & $(0,0)$ & $(1,0)$ & $(0,1)$ & $(1,1)$\\
	$(1,0)$ & $(1,0)$ & $(0,0)$ & $(1,1)$ & $(0,1)$\\
	$(0,1)$ & $(0,1)$ & $(1,1)$ & $(0,0)$ & $(1,0)$\\
	$(1,1)$ & $(1,1)$ & $(0,1)$ & $(1,0)$ & $(0,0)$
\end{tabular}
\end{center}

We can see that these are the ``same'' groups by identifying as follows.
\begin{gather*}
	1 \leftrightarrow (0,0)\\
	3 \leftrightarrow (1,0)\\
	5 \leftrightarrow (0,1)\\
	7 \leftrightarrow (1,1)
\end{gather*}

\begin{definition}
	Let $(G,*)$ and $(H,\cdot)$ be groups. Then a \textbf{group homomorphism} is a function $f\colon G\to H$ such that
	$$f(a*b)=f(a)\cdot f(b),$$
	for any $a,b\in G$.
\end{definition}

\begin{proposition}
	Let $f\colon G\to H$ be a homomorphism.
	\begin{enumerate}
		\item $f(e_G)=e_H$
		\item $f(a^{-1})=f(a)^{-1}$
		\item If $G_1\subseteq G$ is a subgroup, then $f(G_1)=\{f(g)\mid g\in G_1\}\subseteq H$ is a subgroup
		\item If $H_1\subseteq$ is a subgroup, then $f^{-1}(H_1)=\{g\in g\mid f(g)\in H_1\}\subset G$ is a subgroup
	\end{enumerate}
\end{proposition}

\begin{definition}[kernel, image]
	Let $f\colon G\to H$ be a homomorphism. The \textbf{kernel} of $f$ is the set
	$$\ker f=\{g\in G\mid f(g)=e_H\}.$$
	The \textbf{image} of $f$ is the set
	$$\im f=\{f(g)\mid g\in G\}\subset H.$$
\end{definition}

\begin{proposition}
Let $f\colon G\to H$ be a homomorphism.
	\begin{enumerate}
		\item $\ker f$ is a normal subgroup of $G$.
		\item $\ker f=\{e_G\}$ if and only if $f$ is injective.
		\item $\im f$ is a subgroup of H.
	\end{enumerate}
\end{proposition}

\begin{definition}[group isomorphism]
	A group homomorphism $f\colon G\to H$ is called a \textbf{group isomorphism} if $f$ is both injective and surjective, i.e. $f$ is a bijection. If there exists such a homomorphism between $G$ and $H$, we say that $G$ and $H$ are \textbf{isomorphic}, and we write $G\cong H$.
\end{definition}

\begin{example}
	As seen previously, $U(8)\cong \mathbb Z_2\times Z_2$ and the isomorphism $f$ is the identification we had made
\end{example}

So in our motivating question, when we say ``distinct'', we mean up to isomorphism.

\begin{example}
	$\mathbb Z_4\not\cong\mathbb Z_2\times Z_2$ because the group $\mathbb Z_4$ has an element of order 4 and $\mathbb Z_2\times Z_2$ has no such element.
\end{example}

\begin{theorem}[First Isomorphism Theorem]
	Let $f\colon G\to H$ be a group homomorphism. Then
	$$G/\ker f\cong \im f\subset H.$$
\end{theorem}

Let's refine our motivating question: For each integer $n\geq 1$, list all groups $G$ with $|G|=n$ such that any group of order $n$ is isomorphic to one group in the list.

\begin{example}
	Suppose that $p$ is prime. If $|G|=p$, then $G\cong\mathbb Z_p$.

	\begin{proof}
		From the Lecture 1, if $|G|=p$, we proved that $G$ is cyclic, i.e. $G=\langle a\rangle$ for some $a\in G$. So $G=\{a^0=e,a^1,\hdots,a^{p-1}\}$.

		Define a map $\phi\colon G\to\mathbb Z_p$ by $\phi(a^i)=i$. This is clearly a bijection. It is also a homomorphism since, if $i+j=k\pmod{p}$, $\phi(a^ia^j)=\phi(a^k)=k=i+j=\phi(a^i)+\phi(a^j)$.
	\end{proof}
\end{example}

\begin{proposition}
	There is only one group of order $p$ prime up to isomorphism: $\mathbb Z_p$
\end{proposition}

Let's look at cases for small $n$.

\begin{center}
\begin{tabular}{l | l}
	$n$ & all non-isomorphic groups\\ \hline
	1 & $\{0\}$\\
	2 & $\mathbb Z_2$\\
	3 & $\mathbb Z_3$\\
	4 & $\mathbb Z_4$, $\mathbb Z_2\times\mathbb Z_2$\\
	5 & $\mathbb Z_5$\\
	6 & $\mathbb Z_2\times\mathbb Z_3$, $S_3$\\
	7 & $\mathbb Z_7$\\
	8 & $\mathbb Z_2\times\mathbb Z_2\times\mathbb Z_2$, $\mathbb Z_8$, $D_4$, $Q_8$, $\mathbb Z_4\times\mathbb Z_2$
\end{tabular}
\end{center}

\begin{example}
	If $G$ is cyclic, and $|G|=n$, then $G\cong\mathbb Z_n$.
\end{example}

\begin{theorem}
	If $\gcd(m,n)=1$, then $\mathbb Z_{mn}=\mathbb Z_m\times\mathbb Z_n$.
\end{theorem}

\begin{example}
	$\mathbb Z_6\cong\mathbb Z_2\times\mathbb Z_3$ and $\mathbb Z_{12}=\mathbb Z_{2^2}\times\mathbb Z_3$.

	As a counterexample, where $\gcd(m,n)\neq 1$, 
\end{example}

\begin{theorem}[Fundamental Theorem of Finite Abelian Groups]
	Every finite abelian group is isomorphic to a direct product of cyclic groups of prime power order, i.e. of the form
	$$\mathbb Z_{p_1^{d_1}}\times \mathbb Z_{p_2^{d_2}}\times\cdots\times \mathbb Z_{p_r^{d_r}},$$
	where the $p_i$'s are not necessarily distinct.
\end{theorem}

We can thus answer our refined motivating question.

\begin{example}
	Write all non-isomorphic abelian groups of order 100.

	\begin{solution}
		Observe that
		\begin{align*}
			100&=2^2\cdot 5^2\\
			&=2^1\cdot 2^1\cdot 5^2\\
			&=2^1\cdot 2^1\cdot 5^1\cdot 5^1\\
			&=2^2\cdot 5^1\cdot 5^1.
		\end{align*}
		This gives the following groups:
		\begin{itemize}
			\item $\mathbb Z_{2^2}\times\mathbb Z_{5^2}$
			\item $\mathbb Z_2\times\mathbb Z_2\times\mathbb Z_5$
			\item $\mathbb Z_2\times \mathbb Z_2\times \mathbb Z_5\times \mathbb Z_5$
			\item $\mathbb Z_{2^2}\times \mathbb Z_5\times \mathbb Z_5$
		\end{itemize}
	\end{solution}
\end{example}

\begin{corollary}
	If $n$ is square-free, i.e. $n=p_1^1p_2^1\cdots p_r^1$, then there is only one abelian group of order $n$ up to isomorphism, notably,
	$$\mathbb Z_{p_1}\times\cdots\times\mathbb Z_{p_r}.$$
\end{corollary}

\begin{example}
	$\mathbb Z_{15}\cong\mathbb Z_3\times\mathbb Z_5$ is the only abelian group of order 15.
\end{example}