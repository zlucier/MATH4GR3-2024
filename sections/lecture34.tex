\section{Lecture 34 — Geometric Constructions II}

Recall that a number $\alpha\in\mathbb R$ is constructible if we can construct a line segment of length $|\alpha|$ using only straightedge and compass operations.

The set
$$\{\alpha\in\mathbb R\mid\text{$\alpha$ is constructible}\}\subseteq\mathbb R,$$
Is a proper subfield of $\mathbb R$ and a proper extension of $\mathbb Q$.

\begin{definition}[constructible]
	A point $P=(a,b)$ is \textbf{constructible} if both $a$ and $b$ are constructible.
\end{definition}

\begin{lemma}
	Let $F$ be a subfield of $\mathbb R$.
	\begin{enumerate}
		\item If a line $L$ contains points $P_1$ and $P_2$ in $F$, then its equation has the form
		$$ax+by+c=0,\qquad\text{with $a,b,c\in F$.}$$
		\item If a circle has center $P$ in $F$ and a radius $r\in F$, then its equation has the form
		$$x^2+y^2+dx+ey+f=0,\qquad\text{with $d,e,f\in F$.}$$
	\end{enumerate}
\end{lemma}

\begin{proof}
	We prove (b).

	Let $S$ be a circle with center $P=(a,b)\in F^2$ and radius $r\in F$. Then its equation is
	\begin{align*}
		(x-a)^2+(y-b)^2&=r^2\\
		x^2-2ax+a^2+y^2-2by+b^2&=r^2\\
		x^2+y^2+(-2a)x+(-2b)y+(a^2+b^2+r^2)&=0.
	\end{align*}
	But since $a,b,r\in F$, it must also be that $(-2a),(-2b),(a^2+b^2+r^2)\in F$.
\end{proof}

Starting with a field $F$ of constructible numbers, recall how we can add ``new'' points:

\begin{enumerate}[label={\color{main}\bfseries\sffamily (\roman*)}]
	\item Intersect two lines, each of which passes through two points with coordinates in $F$.
	\item Intersect a line and a circle with center and radius in $F$.
	\item Intersect two circles with centers and radii in $F$.
\end{enumerate}

Note that case (iii) reduces to case (ii).

\begin{center}
\begin{tikzpicture}[scale=0.5]
\begin{scope}
	\draw (-1,0) circle (2);
	\draw (1,0) circle (2);
	\node[circle,fill=main,scale=0.5] at (0,{sqrt(3)}) {};
	\node[circle,fill=main,scale=0.5] at (0,{-sqrt(3)}) {};
\end{scope}
\begin{scope}[shift={(8,0)}]
	\draw (-1,0) circle (2);
	\draw (1,0) circle (2);
	\draw (0,2.5) -- (0,-2.5);
	\node[circle,fill=main,scale=0.5] at (0,{sqrt(3)}) {};
	\node[circle,fill=main,scale=0.5] at (0,{-sqrt(3)}) {};
\end{scope}
\begin{scope}[shift={(16,0)}]
	\draw (-1,0) circle (2);
	\draw[opacity=0.1] (1,0) circle (2);
	\draw (0,2.5) -- (0,-2.5);
	\node[circle,fill=main,scale=0.5] at (0,{sqrt(3)}) {};
	\node[circle,fill=main,scale=0.5] at (0,{-sqrt(3)}) {};
\end{scope}
\end{tikzpicture}
\end{center}

In case (i), since the two equations have the form $ax+by+c=0$ with coefficients in $F$, the intersection will have coordinates in $F$.

In case (ii), we want to solve
$$\begin{cases}
	ax+by+c=0,\\
	x^2+y^2+dx+ey+f=0,
\end{cases}$$
where $a,b,c,d,e,f\in F$. Solve $y=-\frac abx-\frac cb$ and substitute into the second equation and obtain
$$x^2+\left(-\frac abx-\frac cb\right)^2+dx+e\left(-\frac abx-\frac cb\right)+f=0.$$
Expanding gives an equation of the form
$$Ax^2+Bx+C=0,$$
where $A,B,C\in F$. So
$$x=\frac{-B\pm\sqrt{B^2-4AC}}{2A}\quad\text{and}\quad y=-\frac ab\left(\frac{-B\pm\sqrt{B^2-4AC}}{2A}\right)-\frac cb.$$
But notice that it is not necessarily the case that these solutions $x,y$ lie in $F$. But $x,y\in F(\sqrt\alpha)$ where $\alpha=B^2-4AC$. So this implies that $\sqrt\alpha$ is constructible.

To recap, when creating points by intersecting two lines, we get points with coordinates in $F$, but when creating points by intersecting a circle and a line, we get points with coordinates in $F(\sqrt\alpha)$, for some $\alpha$ and observe that $[F(\sqrt\alpha):F]=1$ or $[F(\sqrt\alpha):F]=2$. So $F(\alpha)$ is at most a \textbf{quadratic extension}.

\begin{theorem}
	$\alpha\in\mathbb R$ is constructible if and only if there is a sequence of fields
	$$\mathbb Q=F_0\subseteq F_1\subseteq F_2\subseteq\cdots\subseteq F_k$$
	such that $F_i=F_{i-1}(\sqrt\alpha_i)$ for some $\alpha_i\in F_{i-1}$. In particular,
	$$[\mathbb Q(\alpha):\mathbb Q]=2^k,$$
	if and only if $\alpha$ is constructible.
\end{theorem}

\subsection*{Impossible Constructions}

\begin{enumerate}[label={\bfseries\sffamily\color{main}(\arabic*)}]
	\item \textbf{Squaring a circle:} \textit{Given a circle of radius 1, construct a square with the same area as the circle.}

	This problem is equivalent to constructing a square with area $\pi$ and thus segments of length $\sqrt\pi$. So this would require $\sqrt\pi$ to be constructible. But $[\mathbb Q(\sqrt{\pi}):\mathbb Q]=\infty$ since $\pi$, and thus $\sqrt\pi$, is transcendental.

	\item \textbf{Doubling a cube:} \textit{Given a cube of volume 1, construct a cube with double its volume.}

	This problem is equivalent to constructing a cube with side lengths $\sqrt[3]{2}$. But $\sqrt[3]{2}$ is not constructible since
	$$[\mathbb Q(\sqrt[3]{2}):\mathbb Q]=3\neq2^k.$$

	\item \textbf{Trisect an angle:} \textit{Given an angle with measure $\theta$, construct an angle with measure $\theta/3$.}

	Note that the point $\left(\frac 12,\frac{\sqrt 3}{2}\right)$ is constructible. This is the point obtained by intersecting the unit circle with a line forming an angle of $60^\circ$ with the $x$-axis. A point $R$ on the unit circle trisecting this angle would construct $\cos 20^\circ$.

	\begin{center}
	\begin{tikzpicture}[scale=2]
		\footnotesize
		\draw[dashed,main] ({cos(20)},0) -- (20:1);

		\draw[gray] (0,0) -- (60:1.5);
		\draw[gray] (0.15,0) arc (0:60:0.15);

		\draw[main] (0,0) -- (20:1.5);
		\draw[main] (0.35,0) arc (0:20:0.35);

		\draw[->] (-0.25,0) -- (1.5,0);
		\draw[->] (0,-0.25) -- (0,1.5);
		\draw (1,0) arc (0:90:1);
		\node[gray,circle,scale=0.5,fill,label=right:{$\left(\frac 12,\frac{\sqrt 3}{2}\right)$}] at (0.5,{sqrt(3)/2}) {};
		\node[main,circle,scale=0.5,fill,label=right:{$R$}] at (20:1) {};
		\node[below,color=main] at ({cos(20)},0) {$\cos 20^\circ$};

		\node[above right,gray] at (30:0.1) {$60^\circ$};
		\node[right,main] at (10:0.35) {$20^\circ$};
	\end{tikzpicture}
	\end{center}

	Using trigonometric identities,
	\begin{align*}
		\cos 3\theta&=\cos(2\theta+\theta)\\
		&=\cos(2\theta)\cos\theta-\sin(2\theta)\sin\theta\\
		&=(2\cos^2\theta-1)\cos\theta-2\sin^2\theta\cos\theta\\
		&=(2\cos^2\theta-1)\cos\theta-2(1-\cos^2\theta)\cos\theta\\
		&=4\cos^3\theta-3\cos\theta.
	\end{align*}
	So $\cos 20^\circ$ satisfies
	\begin{align*}
		4(\cos 20^\circ)^3-3(\cos 20^\circ)-\cos(60^\circ)&=0\\
		4(\cos 20^\circ)^3-3(\cos 20^\circ)-\frac 12&=0.
	\end{align*}
	Thus, $\cos 20^\circ$ is a root of $p(x)=4x^3-3x-\frac 12$. But $p(x)$ is irreducible over $\mathbb Q$. So $[Q(\cos^\circ):\mathbb Q]=3\neq 2^k$.
\end{enumerate}